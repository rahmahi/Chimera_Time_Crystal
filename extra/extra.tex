\documentclass[a4paper,11pt]{article}
\usepackage[margin=1in]{geometry} 
\usepackage{amsmath}
\usepackage{amsthm}
\usepackage{amssymb}
\usepackage[sort&compress]{natbib}
\usepackage{ulem}
\usepackage{url}
\usepackage{hyperref}
\usepackage{bbm}
\usepackage{dcolumn}

\usepackage[dvipsnames]{xcolor}
\newcommand{\red}[1]{\textcolor{red}{#1}}
\newcommand{\blue}[1]{\textcolor{blue}{#1}}
\usepackage{cancel}


% Author info
\title{Chimera: Short review}
%\author{\small{Mahbub}}
%\date{\small{\today}}

\begin{document}
\begin{center}
	{\large\bf \blue{Chimera: Short review}}\\
	{\tiny \bf Mahbub Rahaman}
\end{center}

The Chimera, as described in Greek mythology, is a hybrid creature that combines the characteristics of a lion, goat, and snake.  Kuramoto and Battogtokh noticed this particularity in a phase oscillator model\cite{kuramoto_coexistence_2002} in the practical world in 2002. They found that when suitable conditions applied to an array of identical oscillators, it is possible to observe a set of mutually synchronized oscillator with a unique frequency and another set of oscillators which are desynchronized with distributed frequencies. Thus there exists two different dynamics in two domain of the oscillator array.
Latter, there were several improvement in the study on Chimera and these are mostly in classical regime where coupled identical oscillators are investigated.

A quantum analog of chimera was proposed by Singh et al.(2011) in (epl)\cite{singh_chimera_2011}. It was the first time, they reported a chimera in a quantum system, specifically in an Ising spin system where strongly and weakly ordered spins coexists simultaneously. The phenomena was observed by magnitude of magnetization in spin system. A few years latter, Bastidas proposed another model comprised of a ring of coupled van der Pol oscillators for investigation over the quantum fluctuation about the semi classical trajectory\cite{bastidas_quantum_2015}. The chimera was detected for the system via  bosonic squeezing, weighted quantum correlations, and measures of mutual information. In the same year, Viennot et al. proposed another chimera spin model\cite{VIENNOT2016678}, where a highly entangled and a totally not entangled regions in the spin system coexits. They found that quantum chimera state is more stable than classical chimera. 

Recently, Sakurai et al. presented a chimera which is comprised of coexisting a discrete time crystal(DTC) and a disorder induced MBL in an one dimensional spin-1/2 network\cite{sakurai_chimera}. The chimera order is observed from the temporal variation of magnetization observable for two distinct regions. They introduced spin rotational error which consequently controls the coupling between the two regions. The entanglement entropy corresponding to each regions in total is found to lie in between thermal and localized state. The sustainability of MBL protects the stability of Chimera.

The quantum chimera can be used in quantum-information studies. The information can be transported from one region to another region via coupling of the spins\cite{VIENNOT2016678}. 
\vskip 1cm
\begin{center}
	{\large\bf \blue{Conclusion ; Chimera manuscript}}\\
	{\tiny \bf MR AS AR}
\end{center}

A quantum chimera state has been discovered, wherein a Discrete Time Crystal (DTC) and a Dynamically Many-Body Localized (DMBL) phase coexist on a one-dimensional \red{disorder free} spin-$1/2$ chain. This was accomplished  by applying dynamical spin-flips regionally (with two regions labeled `A' and `B'), followed by dynamical localization with a high-frequency periodic drive tuned to CDT/DL points determined by particular values of the drive parameters. The findings from numerical simulations suggest that the DTC does not exhibit long-term persistence when weak spin coupling is present, especially in the case of all-to-all interaction ranges. The presence of a robust spin coupling hinders the relaxation of spins away from the desired DTC, stabilizing the phase regionally for long times throughout all interaction ranges. The stability of the DTC in region A is contingent upon the stability of the DMBL phase in region B. The long-range spin interaction is very robust against the exchange energy per spin, and enable the chimera to persist for both weak and strong coupling. We tuned the spin rotational error at region A and found that the chimera can persist only at small rotational error $\epsilon_A$. The inter-region interaction term in the effective Floquet Hamiltonian in Eq.~\ref{eq:floq_eff3} is directly proportional to the spin rotation error denoted as $\epsilon_A$. In the case of a significant value of $\epsilon_A$, there is an increased level of coupling between the two regions. This heightened coupling results in the immediate melting of the DTC for all interactions, except for those that are long-range in nature.

To further study the resilience of this chimera against static perturbations, an additional external static field was included. Numerical simulations have demonstrated that the system is robust in the face of minor perturbations, as evidenced by the sustained presence of the DTC phase inside region A over an extended duration. However, when subjected to more significant perturbations, the system's robustness diminishes, leading to rapid dissolution of the DTC phase across all spin-interaction types. The Floquet Hamiltonian, obtained analytically, illustrates the occurrence of period-doubling sub-harmonic solutions to the Schr\"odinger dynamics. These solutions arise specifically at the reversal of global ferromagnetic order in the spin chain. This conclusion is substantiated by exact simulations. The entanglement entropy rises at a very slow rate for both weak and strong regional spin interactions. Hence, the entire system is effectively prevented from undergoing thermalization, even in cases where the DTC phase disintegrates inside region A. The aforementioned property can be employed as quantum memory in various applications, as well as in the development of exceptionally accurate quantum clocks. The experimental realization of the suggested model and dynamics can be achieved using trapped ions, as demonstrated in previous studies ~\cite{sakurai_phys_nodate, Friedenauer2008}.

\sout{To summarize, we and examined the various components that contribute to the stability of the DTC-DMBL-chimera order. Long-range spin interactions have been identified as the optimal choice for constructing the chimera.} The scope of our analysis encompasses various magnitudes of power-law decay in spin-spin contacts, the magnitude of regional spin rotational error, the intensity of spin-coupling interactions, the presence of an extra external static field, and the parameters associated with the periodic drive. The aforementioned contributions play a key role in ensuring the stability of the chimera order. \red{We expect it will unveil a new set of quantum-chimera possibilities.}

\bibliographystyle{unsrt}
\bibliography{chmr_revw}
\end{document}
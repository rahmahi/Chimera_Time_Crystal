% TeX'ing this file requires that you have AMS-LaTeX 2.0 installed
% as well as the rest of the prerequisites for REVTeX 4.0
%
% See the REVTeX 4 README file
% It also requires running BibTeX. The commands are as follows:
%
%  1)  latex apssamp.tex
%  2)  bibtex apssamp
%  3)  latex apssamp.tex
%  4)  latex apssamp.tex
%
%\documentclass[prb,showkeys,preprintnumbers,amsmath,amssymb, 11pt]{revtex4}
%\documentclass[preprint,showpacs,showkeys,preprintnumbers,amsmath,amssymb]{revtex4}

% Some other (several out of many) possibilities
%\documentclass[preprint,aps]{revtex4}
%\documentclass[aps, two column, amsmath,amssymb,floatfix]{revtex4}
%\documentclass[showkeys,showpacs,amsmath,amssymb,onecolumn,superscriptaddress,prl]{revtex4-1}% Physical Review B  

%\documentclass[aps,prl,reprint,showpacs,floatfix,superscriptaddress, onecolumn, 9pt]{revtex4-2}

\documentclass[aps,prb,reprint,showpacs,floatfix,superscriptaddress, onecolumn, 9pt]{revtex4-2}

\usepackage{amsmath,amsthm,amssymb}
\usepackage{graphicx}% Include figure files
\usepackage{dcolumn}% Align table columns on decimal point
\usepackage{bm}% bold math
\usepackage{color}
\usepackage{epsfig}
\usepackage{multirow}
\usepackage{mathrsfs}
\usepackage{hyperref}
\usepackage{cleveref}
\usepackage{epstopdf}
\usepackage{subfigure}
\usepackage{autobreak}

\usepackage{physics}
\usepackage{bbm}

%Macros for mathematical notations

\newcommand{\V}[1]{\boldsymbol{#1}} %# vector
\newcommand{\M}[1]{\boldsymbol{#1}} %# matrix
\newcommand{\Set}[1]{\mathbb{#1}} %# set
\newcommand{\D}[1]{\Delta#1} %# \D{t} for time step size
\renewcommand{\d}[1]{\delta#1} %# \d{t} for small increment
\newcommand{\av}[1]{\left\langle #1\right\rangle } %take average

\newcommand{\sM}[1]{\M{\mathcal{#1}}} %matrix in mathcal font
\newcommand{\dprime}{\prime\prime} % double prime
%\global\long\def\i{\iota}
%\renewcommand{\i}{\iota} %i for imaginary unit
%\renewcommand{\i}{\mathsf i} %i for imaginary unit
\newcommand{\follows}{\quad\Rightarrow\quad} %=>
\newcommand{\eqd}{\overset{d}{=}} %=^d
\newcommand{\spe}[1]{\mathscr{#1}}  %important quantities in mathscr font
\newcommand{\eps}{\epsilon}

\newcommand{\ar}[1]{{\color{blue}#1}} % for authors' response


\begin{document}
\preprint{Preprint}

\title{Response to referees’ comments for the manuscript (BJ14850) - Time crystal embodies chimera in a periodically driven quantum spin system}
\author{}
% \date{}

\maketitle

\noindent Dear Editor,
\vskip 0.5cm
We thank the referees for carefully evaluating our manuscript, for their positive assessment and suggestions, and for making several important remarks that have helped us to improve our presentation. We comment on their remarks in detail below.

%\vspace{1em}
%\noindent \textbf{Major Changes}

%\ar{
%\begin{enumerate}
%    \item 
%    \item 
%\end{enumerate}
%}

\vspace{1em}

\noindent \textbf{Response to First Referee}

\begin{enumerate}
    \item The referee says, ``\textit{The paper is littered with typographical errors and poorly worded sentences, which make it hard to read and understand.}"\\

    \ar{
We sincerely regret and thank the referee for stressing the importance of properly correcting any errors and improving sentence structure. We have completed this task to the best of our abilities. In addition, we improved several of our figures and restructured sections of the material to improve clarity.
    }
    \item The referee says, ``\textit{The mathematical
    expansion used made in going from Eq 6 to Eq 7, which seems central, makes utterly no sense in the supposedly high-frequency limit expansion being used. You'd need 4 h/omega to be small to truncate the Jacobi-Anger sum. But you need 4h/omega to be O(1) to lie at a root of J$_0$. So the approximation made is inconsistent with the regime in which the expression 7 is being used.}".\\

    \ar{
We appreciate the referee bringing this up. We believe the question occurred as a result of a lack of clarity in our derivation. As a result, we have corrected and revised the writing in ''Section II. Interacting Dynamical Localization" as well as ''Appendix A," and clarified how we obtain the optimal values of the drive parameters amplitude $h$ and frequency $\omega$ that lead to localization during the $T_2-$cycle. \\

    The moving frame Hamiltonian is defined in Eq.(6) in the manuscript as
    \begin{equation}
        \hat{H}^{mov}(t) = \hbar\sum_{ij} J_{ij} \Big(\hat{\sigma}^y_i\hat{\sigma}^y_j\Big) e^{i 2\zeta(t)\hat{\sigma}^z_i}  e^{i 2\zeta(t) \hat{\sigma}^z_j},
        \label{eq:hmov}
    \end{equation}
    where $\zeta(t)\equiv \frac{h}{\omega}[1-\cos(\omega t)]$. Now, before we apply the Jacobi-Anger expansion to this expression, let us define $ A(\eta,\theta)\equiv\displaystyle e^{i\eta \cos(\theta)} = \sum_{n=-\infty}^{\infty} \mathcal{J}_n(\eta)e^{in\theta} $. In this case, $A$ is a function of two independent variables, $\eta$ and $\theta$, and $0\leq\theta\leq 2\pi$. The coarse-grained average $\bar{A}(\eta)$ is defined as the average of $A(\eta, \omega)$ across all values of $\theta$. Thus,
    \begin{equation}
    \bar{A}(\eta) \equiv \expval{e^{i\eta\cos(\theta)}}_{\theta} = \sum_{n=-\infty}^{\infty} \mathcal{J}_n(\eta) \expval{e^{in\theta}}_{\theta} =\mathcal{J}_0(\eta).
    \end{equation}
In the paper, we also have two independent parameters, the driving frequency $\omega$ and amplitude $h$, which are connected to $\eta, \theta$ by a trivial linear map, $\eta=4h/\omega, \theta=\omega t$. Thus, we can obtain the coarse-grained average in a similar manner to yield
    \begin{equation}
        \bar{A}(h,\omega) \equiv \expval{e^{i\frac{4h}{\omega}\cos(\omega t)}}_t = \sum_n\mathcal{J}_n\left(\frac{4h}{\omega}\right)\expval{e^{i n \omega t}}_t = \mathcal{J}_0(\eta) ;  \forall{\eta}\in\mathbb{R},
        \label{eq:jacang}
    \end{equation}
where our coarse-graining is over time scales $t\gtrsim 2\pi/\omega$, allowing us to average all harmonic oscillations to zero (regardless of the value of the amplitude), leaving behind only the constant $n=0$ term in the Fourier series on the RHS. The Rotating Wave Approximation (RWA) states that if $\omega \gg 1$, then  $\bar{A}(h,\omega)$ approximates $A(h, \omega t)$. The RWA does not impose any constraint on $h$.
    This allows us to independently alter $\omega \gg 1$ \emph{and} $h \gg 1$ so that the ratio $\frac{4h}{\omega} \approx \order{1}$ without violating the RWA criteria. For example, if $\omega=100$ and $h\approx 60.1206\dots$, we get $\bar{A}(h,\omega)=\mathcal{J}_0(2.404\dots)=0$, and, because the RWA still holds due to the huge $\omega$, we may approximate $A(h,\omega t)$ by $0$.

    Therefore, it is reasonable to approximate   $\hat{H}^{mov}$ in our manuscript by 
    \begin{align}
		\hat{H}^{_{RWA}} = \hbar\sum_{ij} J_{ij} \Big(\hat{\sigma}^y_i\hat{\sigma}^y_j\Big) \mathcal{J}_0\Big(\frac{4h}{\omega}\Big)\Bigg[\mathcal{J}_0\Big(\frac{4h}{\omega}\Big) + \cos(\frac{4h}{\omega}) -\hat{\sigma}^z_i\hat{\sigma}^z_j \Bigg\{\mathcal{J}_0\Big(\frac{4h}{\omega}\Big) - \cos(\frac{4h}{\omega})\Bigg\} + \frac{i}{2} (\hat{\sigma}^z_i + \hat{\sigma}^z_j) \sin(\frac{4h}{\omega})\Bigg],
	\end{align}
    as long as $\omega\gg 1$ (say $\sim 10^2$). If we now adjust $h$ (also say $\gtrsim 10^2$) so that ($4h/\omega$) lies on a root of $\mathcal{J}_0(4h/\omega)$, then $\hat{H}^{_{RWA}}$ vanishes, resulting in full localization of the spin system during the $T_2$ cycles.

     We can strengthen our argument with numerical simulations of a simple driven TLS  where RWA is widely known to apply (and this phenomenon referred to as \textit{Coherent Destruction of Tunneling}~\cite{Grossmann1991, Ashhab2007}, or CDT) . Here, the Hamiltonian $\hat{\mathcal{H}}(t) = \hat{H}_0 + \hat{H}_1(t)$, and  
    \begin{align}
    \hat{H_0} &= \Delta \hat{\sigma}^x,\nonumber \\
    \hat{H_1}(t) &= h \cos(\omega t)\hat{\sigma}^z.
    \end{align}}
 \begin{figure}[t!]
	\begin{center}
		\includegraphics[height=9.5cm]{rwa_vs_exact_w_low_n_high_frz_nfrz.pdf}
	\end{center}
	\caption[] {Comparison of the exact time evolution and the RWA-approximaated time evolution of the magnetization $\expval{\hat{\sigma}^z}(t)$ for the harmonically driven quantum two-level system (TLS). The evolution is performed from the eigenstate of $\sigma^z$ at $t=0$ for different drive frequencies $\omega$ with amplitudes $h$. The latter are set to values where $\eta=2h/\omega$ lies either at the first root of $\mathcal{J}_0(\eta)$ (top panels), or a little away from it (bottom panels). The plots are presented in increasing order of $\omega$ from the left to right panels.}
	\label{Fig:compare_exact_rwa}
\end{figure}
    \ar{In the rotating frame obtained from the propagator $\hat{U}(t) = \exp\big[-i \frac{h}{\omega} \sin(\omega t)\hat{\sigma}^z\big]$, the moving frame Hamiltonian is
    \begin{align}
    \tilde{\mathcal{H}}(t)^{mov} &=\hat{U}^\dagger(t) \mathcal{H}(t) \hat{U}(t) - i \hat{U}^\dagger(t) \partial_t \hat{U}(t)\nonumber\\
    &= \hat{U}^\dagger(t) H_0 \hat{U}(t)\nonumber\\
    &= e^{i \frac{h}{\omega} \sin(\omega t)\hat{\sigma}^z} \Big(\Delta \hat{\sigma}^x\Big) e^{-i \frac{h}{\omega} \sin(\omega t)\hat{\sigma}^z}
    \end{align}
    Let us define $\phi \equiv \frac{2h}{\omega} \sin(\omega t)$. Recalling $\hat{S}^{x,y,z} = \frac12 \hat{\sigma}^{x,y,z}$, we can expand the moving frame Hamiltonian as follows.
    \begin{align}
    \tilde{\mathcal{H}}(t)^{mov} &= e^{i \frac{2h}{\omega} \sin(\omega t)\hat{S}^z} \Big(2\Delta \hat{S}^x\Big) e^{-i \frac{2h}{\omega} \sin(\omega t)\hat{S}^z}\nonumber\\
    &= 2\Delta \Big(\hat{S}^x \cos{\phi} - \hat{S}^y \sin{\phi}\Big)\nonumber\\
    &= 2\Delta \bigg\{\hat{S}^x\cos\Big[\frac{2h}{\omega}\sin(\omega t)\Big] - \hat{S}^y\sin\Big[\frac{2h}{\omega}\sin(\omega t)\Big]\bigg\},
    \end{align}
    where we have use the Baker-Campbell Hausdorff formula, together with the angular momentum commutation relations $[\hat{S}^\mu, \hat{S}^\nu]=\epsilon_{\mu\nu\alpha}\hat{S}^\alpha$
    Next, using Jacobi Anger formula, we get
    \begin{equation}
    \tilde{\mathcal{H}}(t)^{mov} = \Delta \Bigg[\hat{\sigma}^x \bigg\{\mathcal{J}_0\Big(\frac{2h}{\omega}\Big) + 2 \sum_{n=1}^{\infty} \mathcal{J}_{2n-1}\Big(\frac{2h}{\omega}\Big)\cos(2n\omega t)\bigg\} -\hat{\sigma}^y\bigg\{2\sum_{n=1}^{\infty} \mathcal{J}_{2n-1}\Big(\frac{2h}{\omega}\Big) \sin\left[(2n-1)\omega t\right]\bigg\}\Bigg]
    \label{eq:hexact}
    \end{equation}
    Now, applying RWA for $\omega \gg 1$ simply involves neglecting the fast-oscillating terms. This yields
    \begin{equation}
    \tilde{\mathcal{H}}(t)^{mov}\approx\hat{\mathcal{H}}^{_{RWA}} = \Delta \mathcal{J}_0 \Big(\frac{2h}{\omega}\Big) \hat{\sigma}^x,
    \end{equation}
with the only constraint being that $\omega \gg 1$. As can be readily seen, the moving frame Hamiltonian almost vanishes when $\omega\gg 1, h\gg 1$, and $2h/\omega$ lies at a root of the zeroth Bessel function.

We have numerically evolved the exact Hamiltonian $\hat{\mathcal{H}}(t)$, as well as the RWA Hamiltonian $\hat{\mathcal{H}}^{_{RWA}}$, for multiple sets of drive parameters, both at and away from values where $\eta=2h/\omega$ lies on a root of $\mathcal{J}_0(\eta)$. We have contrasted their dynamics by comparing $\expval{\hat{\sigma}^z}(t)$, obtained by using the QuTiP 'mesolve' integrator of the Schr\"odinger equation with both Hamiltonians. In both cases, the state of the system was set to the $+1$ eigenstate of $\sigma^z$ at $t=0$. The results are plotted in fig~\ref{Fig:compare_exact_rwa}. The top panels are plots for the case where $\eta$ lies on the first root of $\mathcal{J}_0$, whereas the bottom panels are plots for when it does not. We have considered four different drive frequencies, ranging from small to high values ($\omega = 0.05, 5.0, 10.0, 20.0$) with appropriately adjusted drive amplitudes. The plots are arranged in order of increasing $\omega$ from the left to right panels of fig~\ref{Fig:compare_exact_rwa}. As can be readily seen, the exact and RWA results do not match very well when $\omega \lesssim 1$. However, as $\omega$ increases (together with $h$), they come closer, with slightly better matching away from the root than on it. For either cases, when $\omega=20 \gg 1$, the results match very well.
    
    We hope that our response clarifies the methodology that we have used in our analysis.
    }
\end{enumerate}
  
\noindent \textbf{Response to Second Referee}

\begin{enumerate}
    \item The referee says,``The original work of Bruno on no-go-theorems for time crystals should be mentioned in the introduction: P. Bruno PRL 111, 070402(2013) "\\

    \ar{
    We thank the referee for the suggestion to mention Bruno's no-go theorem in the manuscript. We have introduced Bruno's work in the second paragraph of the Introduction part.
    }
    \item The referee says,``The reference in 34 is incomplete "

    \ar{
    We thank the referee for pointing out the mistake. We have fixed the reference in the bibliography. Due to the inclusion of some more references, this reference is moved to ref no. 37. However, we found that the ``doi" number is unavailable corresponding to Kuramoto's article. Other information viz. author, title, page number, volume number, year of publication, and journal name are included in the bibliography. 
    }

    \item The referee says, ``The time-evolution operator in Eq. 4 should contain a time-ordering operator "

    \ar{
    We thank the referee for detecting the mistake in Eq 4. We have corrected and put down a time-ordering operator in Eq. 4
    }

    \item The referee says, ``The zero of Bessel function "DL condition" naming convention seems questionable. Dynamical localization refers to infinite periodically driven systems. The phenomenon of``coherent destruction of tunneling" (CDT) could be more closely related to the studied case of a few spins. Maybe one should speak of CDT/DL condition or so. In any case, a reference to the original CDT work by F. Grossmann et al, PRL 67,516 (1991) would be in order; see also the work of Kayanuma and Saito in PRA 77, 010101(R) (2008)."\\

    \ar{
    We are deeply thankful to the referee for this comment. We agree with the referee that Coherent Destruction of Tunneling(CDT) is a finite-size phenomenon and Dynamical localization is an infinite-size counterpart of CDT. Our proposed model and the results are valid for all possible sizes of the spin-1/2 chain. Yet in the article, we have presented a few spin version of the model. So we agree with the referee in naming the localization condition and we have replaced ``DL point" with ``CDT/DL point''. We have introduced both Grossmann and Kayanuma's papers in the Introduction part and discussed CDT and DL.
    }

    \item The referee says,``Several times in the ms and the appendix, the Floquet Hamiltonian is mentioned. Is this the usual object, i.e., the Hamiltonian augmented by the time-derivative $(times -i\hbar)$? Then this should be explicitly stated in terms of a formula, e.g. after the statement on page 10:

    ``...through a suitable transformation" ."\\

    \ar{
    We thank the referee for pointing out the mistake. We have introduced the expression $\displaystyle \hat{H}_F(t) = \Bigg(\hat{H}(t) - i\hbar \pdv{t}\Bigg)$ at the exact place suggested by the referee in the manuscript. Due to some other corrections in the manuscript, the expression is moved to page no 11.
    }
    \item The referee says, `` After Eq. (A2) there seems to be a scalar product \begin{verbatim}
    {\hat n \cdot\vec{sigma}}.
    \end{verbatim} The dot indicating scalar product should be moved upwards."\\

    \ar{
    We thank the referee for pointing out the mistake. We have corrected the typographical error on page no. 9 of the manuscript.
    }
\end{enumerate}
\vskip 1cm 
\noindent \textbf{Summary of important changes to the  manuscript}


\begin{enumerate}
    \item We have introduced Bruno's no-go theorem in the second paragraph in the introduction part of the manuscript.
    \item We have replaced the sentence ``Later, Oshikawa and Watanabe proved that the TC cannot exist in a thermal equilibrium system (\textit{the no-go theorem})." with ``Also, Oshikawa and Watanabe proved in another no-go theorem that the TC cannot exist in a thermal equilibrium system" of the second paragraph.
    \item We have introduced a new paragraph in the introduction part right after the second paragraph, and briefly discussed CDT and DL, and mentioned both Grossmann and Kayanuma’s paper.
    \item In ``Section I. the Model and System Dynamics", we have briefly explained how spins in both regions A and B are localized during $T_1$ time cycle(s). We have replaced the sentence ``This keeps the spins localized in region A at each spin-flip operation via quantum tunneling destruction." with ``This makes each of the spins of the system independent of each other, preventing any growth of correlations or entanglement between them. This leads to the spins being localized in their respective regions during $T_1$ time cycle". 
    \item We have introduced a time ordering operator ($\mathcal{T}$) to correct Eq.(4).
    \item We have replaced `$\xi$' in Eq.(5) and (6) with `$\zeta(t)$' to make it consistent with the Appendix derivations.
    \item In the third paragraph of Section II, we have introduced RWA to the moving frame Hamiltonian and explained how the drive parameters $h$ and $\omega$ can be controlled to attain a localization in the second cycle(s) of the drive. We also replaced Eq.(7) with a newer version. We have explained and presented a detailed derivation of Eq.(7) in Appendix A.
    \item In the fourth paragraph of the section II, we have replaced, ``\dots $h$ and $\omega$  are varied in such a way'' with ``\dots $h$ and $\omega$  are controlled in such a way''. 
    
    In the same paragraph, we replaced ``\dots the effective Hamiltonian $\hat{H}^{mov}$ which localizes the system during $T_2$. The Eq.(7) describes the necessary condition for emergence of DMBL during $T_2$.'' with ``\dots $\hat{H}^{_{RWA}}$, hence inducing localization of the system. The Eq.(7) describes the necessary condition for emergence of DMBL during $T_2$ time cycles''.
    \item We have corrected the labeling in Fig.4 in page 5.
    \item In section III page 5, in the last paragraph, we have replaced the denotation `$\zeta$' with `$\chi$' also we replace ``$\dots\zeta(2n+1)\pi$'' with ``$\dots (2n+1)$''.
    \item In Appendix A, we have fixed the denotation of the unitary evolution operator at `$\hat{U}(t)$' and also fixed the expression $\zeta(t) = \frac{h}{\omega}(1-\cos(\omega t))$. 

    We corrected Eq.(A2) by introducing $\mathbbm{1}$.

    We have corrected the typographical error and rewritten the expression, 
    $e^{i a\left(\hat{n} \cdot \vec{\sigma}\right)} = \mathbb{I}\cos{a} + i (\hat{n} \cdot \vec{\sigma}) \sin{a}$, on page 9.

    In the next paragraph right after Eq.(A2), We corrected the explanation and derivation of the RWA Hamiltonian equation and explained it.
    
    \item We have fixed the bibliography regarding Kuramoto's paper.
    \item We have replaced the term ``DL point" with ``CDT/DL point" in the whole manuscript.
    
    \item We have introduced the expression on Floquet Hamiltonian $\displaystyle \hat{H}_F(t) = \hat{H}(t) - i\hbar \pdv{t}$
    \item We have presented the derivation of the Effective Floquet Hamiltonian in Appendix B in detail.
\end{enumerate}

\bibliography{chimera_refs}


\end{document}